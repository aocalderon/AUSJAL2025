\section{Conceptual Framework}

This study is grounded in a political-ecological approach to environmental change, which emphasizes the role of political and economic power structures in shaping ecological outcomes. Specifically, it explores how the occurrence and distribution of wildfires in Argentina are influenced by political alliances and economic interests at the subnational level. The conceptual framework integrates theories of environmental governance, remote sensing science, and political economy.

At the core of this framework is the hypothesis that wildfires are not solely the result of climatic or ecological variables, but are also the outcome of strategic decisions made by actors who may benefit from post-fire land transformation. These actors include pro-business coalitions, agricultural developers, real estate interests, and extractive industries that are often embedded in provincial governance structures. In this context, wildfires can be seen as both environmental events and political outcomes.

Remote sensing provides a unique lens to study this dynamic. Satellite-derived data serve both as a diagnostic tool and an evidentiary basis for linking fire occurrence with patterns of land cover change and land use conversion. The conceptual framework therefore treats remote sensing outputs—such as fire occurrence, burn severity indices (e.g., NBR, dNBR), and LULC transitions—as observable indicators of broader socio-political processes.

The framework also incorporates spatial political analysis by integrating fire data with information on land tenure regimes, infrastructure development, and subnational political configurations. This allows for the identification of spatial correlations between the intensity of wildfires and areas governed by political coalitions with vested economic interests in land use change.

In this way, the framework conceptualizes wildfires not only as ecological disturbances but as socio-politically mediated events. It situates wildfire monitoring within a broader effort to reveal hidden patterns of governance, land dispossession, and environmental policy enforcement (or lack thereof). The combined use of remote sensing and political data thus enables a multi-scalar analysis that bridges satellite observations with ground-level political realities.

Ultimately, this conceptual framework supports the design of an empirical strategy that is both data-rich and theoretically grounded, allowing for the testing of hypotheses about the relationship between governance, economic interest, and environmental transformation in Argentina.
