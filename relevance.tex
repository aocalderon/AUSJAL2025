\section{Relevance and Background}

Remote sensing and satellite image analysis can significantly support political-ecological research by providing independent, spatially explicit, and temporally continuous data that enhance the understanding of wildfire dynamics and their political drivers. There are several reasons that support the use of this kind of technologies among them the precision of the measurements, change detection features, interaction and interoperability, and environmental monitoring and validation.

Remote sensing allows researchers to detect and map wildfires with precision. Sensors such as MODIS, VIIRS, and Landsat help identify burned areas and assess fire extent and severity over time using indices like the Burned Area Reflectance Classification (BARC) and the Normalized Burn Ratio (NBR). These tools also enable the tracking of fire frequency and the identification of persistent fire-prone zones, which may correlate with politically or economically strategic regions.

In addition, satellite imagery is instrumental in monitoring land use and land cover (LULC) change over time. By comparing pre- and post-fire imagery, researchers can analyze transitions such as the conversion of native forests into agricultural or urban areas, often revealing patterns of deforestation in protected or indigenous territories. These insights directly support hypotheses regarding strategic land clearing.

Remote sensing data can also be overlaid with political and economic variables, including land tenure information, infrastructure development (like roads and pipelines), and administrative boundaries. This enables spatial correlation between wildfire occurrence and areas governed by pro-business coalitions or subject to high land value pressures. Similarly, environmental drivers can also be monitored. Vegetation indices such as NDVI and EVI provide information on forest health and fuel load, while drought and climate indicators (e.g., SPI, VCI) help control for natural variability when analyzing human-caused fire patterns.

Without a doubt, satellite data offer a critical tool to validate or challenge ground-based and administrative records. They can confirm or contradict official fire reports, uncover unreported or illegal fires, particularly in remote or politically sensitive areas, and reveal timing mismatches between wildfire events and policy changes, such as the repeal of environmental regulations.

In the context of this study, tools like Google Earth Engine have already been used to verify land clearing after wildfires. Global Forest Watch provides valuable annual datasets on tree cover loss attributable to fire, which are ideal for both mapping and statistical analysis. FIRMS (NASA) delivers near real-time fire detection that can support ongoing fire monitoring and event validation. 

In particular, the use of remote sensing technologies for wildfire detection, mapping, and monitoring has become increasingly central to wildfire science and environmental policy. These techniques offer systematic, cost-effective, and timely data essential for understanding the spatial and temporal dynamics of fires, especially over large or inaccessible areas.

One of the most established uses of remote sensing in wildfire studies is the detection of active fires and the delineation of burned areas. Satellite sensors such as MODIS and VIIRS are widely used for detecting thermal anomalies \cite{giglio2003}. The NASA FIRMS (Fire Information for Resource Management System) project uses these datasets to provide near-real-time fire alerts globally, making them essential for fire response and early warning systems.

Mapping of burned areas and assessing fire severity often relies on spectral indices such as the Normalized Burn Ratio (NBR) and its derivative, the differenced NBR (dNBR), which compare pre- and post-fire reflectance \cite{key2006,miller2007}. Landsat imagery is particularly effective due to its spatial resolution, long temporal record, and open availability \cite{roy2005}.

Remote sensing facilitates the reconstruction of fire histories and regimes, enabling researchers to analyze trends in frequency, intensity, and seasonality over decades. For example, Global Forest Watch’s tree cover loss datasets integrate MODIS and Landsat data to track annual forest disturbances, including fire, from 2001 onwards \cite{hansen2013}. These datasets have been used to document rising fire incidence in the Amazon and Africa, often linked to land-use change and climate variability \cite{andela2017,aragao2014}.

Another key application of satellite data is in monitoring land use and land cover (LULC) changes associated with fires, particularly for analyzing intentional burning and deforestation. Using time series of Landsat or Sentinel-2 imagery, researchers have mapped the transformation of forest to pasture, cropland, or urban areas \cite{fraser2003,pereira2020}.

Advances in sensor fusion --combining datasets from multiple platforms-- have improved fire detection and mapping accuracy. Sentinel-2’s high spatial resolution imagery is often used to refine fire boundaries detected by MODIS, while SAR data from Sentinel-1 can assess vegetation structure in cloud-prone areas \cite{tanase2015}. Emerging technologies, such as cloud-based platforms like Google Earth Engine and machine learning classification methods, are enabling scalable, large-area analyses of wildfire patterns \cite{silva2019,cochrane2003}.

Despite their value, remote sensing tools face challenges such as difficulty detecting low-intensity understory fires, mixed pixel effects, and reliance on ground validation. There is also a trade-off between spatial and temporal resolution, which must be balanced depending on the fire monitoring objective \cite{chuvieco2010}.

Remote sensing thus plays a critical role in detecting, understanding, and responding to wildfires, and it is particularly powerful when integrated with socio-political data in studies investigating the drivers behind fire activity.  Altogether, remote sensing outputs can serve as both dependent variables (e.g., hectares burned) and explanatory layers (e.g., proximity to infrastructure or degree of forest fragmentation), and greatly enhance the analytical depth of case studies, including those focused on provinces like Salta and Córdoba.